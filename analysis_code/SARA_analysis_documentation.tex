% \documentclass[english]{article}
\documentclass[11pt]{article}
%\usepackage[T1]{fontenc}
%\usepackage[latin9]{inputenc}
\usepackage{geometry}
% \geometry{verbose,tmargin=3cm,bmargin=3cm,lmargin=3cm,rmargin=3cm}
\usepackage{color}
\usepackage{amsmath}
\usepackage{amsthm}
\usepackage{setspace}
\onehalfspacing
% \usepackage{babel}
\newcommand{\sam}[1]{\textcolor{blue}{[SAM:\ #1]}}

\begin{document}

\title{Statistical Analysis for SARA}

\author{Tianchen Qian, Mashfiqui Rabbi, Susan Murphy}

\date{2018.09.25}

\maketitle
In this document, we describe the statistical methods to conduct the
primary analysis, secondary analysis, and exploratory analysis for
SARA. They are implemented in the attached R code; usage of the R
code doesn't require understanding of the sections with {*}.

Throughout this document, we assume the treatment probability is a
known constant (constant across time and across all subjects).

The wording of primary hypotheses, secondary hypotheses, and exploratory
analysis are all extracted from Mash's working paper.

\section{Statistical method for primary analysis}

\subsection{Primary hypothesis 1\label{subsec:Primary-hypothesis-1}}

$H_{0}$: The 4pm push notification with inspirational quote will
\textbf{not} increase the full completion of survey and/or active
task the same day as compared to no inspirational quote.

$H_{1}$: The 4pm push notification with inspirational quote will
increase the full completion of survey and/or active task the same
day as compared to no inspirational quote.

\subsubsection{Test statistic and critical value}

\textbf{Data notation for testing this hypothesis.} The data collected
from participant $i$ is denoted by $(Z_{i1},A_{i1},Y_{i1},\ldots,Z_{i,T_i},A_{i,T_i},Y_{i,T_i})$,
where the subscript $1,2,\ldots,T_i$ denotes day in the study. Here we allow the length of observations $T_i$ to be different for each individual. $Y_{it}$
denotes the binary outcome (full completion of either the survey or
completion of both active tasks) on day $t$. $A_{it}$ is the indicator
of treatment on day $t$ ($A_{it}=1$ if sent 4pm push notification,
0 if not). $Z_{it}$ is the control variables on day $t$. The purpose
of $Z_{it}$ is to reduce variance in the analysis. The definition
of which observations can be in $Z_{it}$ depends on the definition
of $A_{it}$: $Z_{it}$ can only include information that is available
prior to $A_{it}$. In this particular example, $Z_{it}$ can only
include information that is available prior to 4pm on day $t$. For
instance, $Z_{it}$ can include variables such as: weather on day
$t$ (before 4pm), previous day adherence $Y_{i,t-1}$, and other
information available before 4pm on day $t$. Let $I_{it}$ denote the availability status indicator for day $t$: if unavailable, the treatment is not delivered.

\textbf{Marginal treatment effect on log scale. }To form the test
statistic, we consider the marginal treatment effect on the log scale:
\begin{align}
\beta_{0} & :=\log\left(\frac{\sum_{t=1}^{T_i}E[Y_{it}|A_{it}=1, I_{it} = 1]}{\sum_{t=1}^{T_i}E[Y_{it}|A_{it}=0, I_{it} = 1]}\right)\nonumber \\
 & =\log\left(\frac{\sum_{t=1}^{T_i}e^{\beta_{0t}}E[Y_{it}|A_{it}=0, I_{it} = 1]}{\sum_{t=1}^{T_i}E[Y_{it}|A_{it}=0, I_{it} = 1]}\right),\label{eq:estimand-primary1}
\end{align}
where $\beta_{0t}=\log\left(\frac{E[Y_{it}|A_{it}=1, I_{it} = 1]}{E[Y_{it}|A_{it}=0, I_{it} = 1]}\right)$.
Thus the definition of $\beta_{0}$ given in (\ref{eq:estimand-primary1})
is marginal both accross time as well as across users.

\textbf{Test statistic and critical value.} In Section \ref{subsec:Technical-details-1},
we describe how to construct an estimator $\hat{\beta}$ for this
marginal treatment effect $\beta_{0}$, and an estimate of its variance
$\widehat{\text{Var}}(\hat{\beta})$. The test statistic $T$ is defined
by
\begin{equation}
T=\frac{\hat{\beta}}{\{\widehat{\text{Var}}(\hat{\beta})\}^{1/2}}.\label{eq:test-primary1}
\end{equation}
To conduct one-sided hypothesis test with significance level $0.025$,
the critical value is $c=t_{n-1-q}^{-1}(1-0.025)$. If $T>c$, we
reject $H_{0}$. Here, $n$ is the sample size, $q$ is the length
of vector $Z_{it}$ (including the added intercept), and $t_{n-1-q}^{-1}(\gamma)$
denotes the $\gamma$-quantile of $t$-distribution with $(n-1-q)$
degrees of freedom. For example, suppose $Z_{it}$ includes two variables:
the previous day adherence $Y_{i,t-1}$ and the weather of day $t$.
Then $q=2+1=3$.

\textbf{Software.} R function to conduct primary hypothesis 1 is
\textsf{SARA\_primary\_hypothesis\_1} in \textsf{SARAanalysis.R}.

\subsubsection{Statistical details{*} \label{subsec:Technical-details-1}}

$\hat{\beta}$ in the test statistic (\ref{eq:test-primary1}) is
obtained by solving the following estimating equation (simutaneously
solving for $\hat{\alpha}$ and $\hat{\beta}$):
\begin{equation}
0=\sum_{i=1}^{n}\sum_{t=1}^{T_i} I_{it} e^{-A_{it}\beta}\left(Y_{it}-e^{Z_{it}^{T}\alpha+A_{it}\beta}\right)\begin{pmatrix}(A_{it}-p_{a})\\
e^{Z_{it}^{T}\alpha}Z_{it}
\end{pmatrix}.\label{eq:ee-primary1}
\end{equation}
Here, $p_{a}=P(A_{it}=1)$ is the randomization probability (assumed
constant across all time and all subjects). The superscript $T$ denotes
matrix transpose. Note that as in linear regression, we add an intercept
1 to the vector $Z_{it}$. It turns out that $\hat{\beta}$ is consistent
for $\beta_{0}$ in (\ref{eq:estimand-primary1}) even if $e^{Z_{it}^{T}\alpha}$
is a wrong model for $E(Y_{it}=1\mid A_{it}=0)$. In order words,
the choice of $Z_{it}$ only affects the variance of $\hat{\beta}$,
but does not affect the consistency of $\hat{\beta}$.

Mathematically, solving (\ref{eq:ee-primary1}) for $\hat{\beta}$
in this primary hypothesis is a special case of the estimating equation for the exploratory analysis
in Section \ref{subsec:Technical-details-3}. Therefore, the formula for
computing $\widehat{\text{Var}}(\hat{\beta})$ is deferred to Section
\ref{sec:var}.

\subsection{Primary hypothesis 2\label{subsec:Primary-hypothesis-2}}

$H_{0}$: Among individuals who complete the survey, providing a post-survey-completion
meme or gif will \textbf{not} yield a higher rate of completion of
the survey or active task the next day than not providing meme/gif
reinforcement after survey completion.

$H_{1}$: Among individuals who complete the survey, providing a post-survey-completion
meme or gif will not yield a higher rate of completion of the survey
or active task the next day than not providing meme/gif reinforcement
after survey completion.

\subsubsection{Test statistic and critical value}

\textbf{Data notation for testing this hypothesis.} The data collected
from participant $i$ is denoted by $(Z_{i1},A_{i1},Y_{i1},\ldots,Z_{i,T_i},A_{i,T_i},Y_{i,T_i})$,
where the subscript $1,2,\ldots,T_i$ denotes day in the study. $Y_{it}$
denotes the binary outcome (full completion of either the survey or
completion of both active tasks) on day $t$. $A_{it}$ is the indicator
of treatment on day $t$ ($A_{it}=1$ if provided a post-survey-completion
meme or gif, 0 if not). $Z_{it}$ is the control variables on day
$t$, which can only include information that is available prior to
$A_{it}$. Let the survey-completion indicator $S_{it}=1$ if participant
$i$ completed the survey on day $t$, and set $S_{it}=0$ otherwise. Let $I_{it}$ denote the availability status indicator for day $t$: if unavailable, the treatment is not delivered. (Two notes: (i) $S_{it} = Y_{it}$. (ii) In the estimating equation we will multiply $S_{it}$ with $I_{it}$ to obtain the ``overall'' availability indicator, and the inference is conditional on these available dates. Here we separate those for conceptual clearness.)

\textbf{Difference from primary hypothesis 1.} (i) In testing this
hypothesis, $Z_{it}$ can include information that is available prior
to the occurrence of post-survey-completion meme on day $t$. So here
(unlike in primary hypothesis 1) we can include in the controls, $Z_{it}$,
data collected by the survey on day $t$. (ii) In testing this hypothesis,
we are interested in the effect of $A_{it}$ on $Y_{i,t+1}$. In using
the software, however, do not recode the outcome manually because
the algorithm already takes this into account.

\textbf{Marginal treatment effect on log scale. }To form the test
statistic, we consider the marginal treatment effect on the log scale
for those who are available:
\begin{align}
\beta_{0} & :=\log\left(\frac{\sum_{t=1}^{T_i - 1}E[Y_{i,t+1}|A_{it}=1,I_{it}=1, S_{it} = 1]}{\sum_{t=1}^{T_i - 1}E[Y_{i,t+1}|A_{it}=0,I_{it}=1, S_{it} = 1]}\right)\nonumber \\
 & =\log\left(\frac{\sum_{t=1}^{T_i - 1}e^{\beta_{0t}}E[Y_{i,t+1}|A_{it}=0,I_{it}=1, S_{it} = 1]}{\sum_{t=1}^{T_i - 1}E[Y_{i,t+1}|A_{it}=0,I_{it}=1, S_{it} = 1]}\right),\label{eq:estimand-primary2}
\end{align}
where $\beta_{0t}=\log\left(\frac{E[Y_{i,t+1}|A_{it}=1,I_{it}=1, S_{it} = 1]}{E[Y_{i,t+1}|A_{it}=0,I_{it}=1, S_{it} = 1]}\right)$.
Thus the definition of $\beta_{0}$ given in (\ref{eq:estimand-primary2})
is marginal both accross time as well as across users.

\textbf{Test statistic and critical value.} In Section \ref{subsec:Technical-details-2},
we describe how to construct an estimator $\hat{\beta}$ for this
marginal treatment effect $\beta_{0}$, and an estimate of its variance
$\widehat{\text{Var}}(\hat{\beta})$. The test statistic $T$ is defined
by
\begin{equation}
T=\frac{\hat{\beta}}{\{\widehat{\text{Var}}(\hat{\beta})\}^{1/2}}.\label{eq:test-primary2}
\end{equation}
To conduct one-sided hypothesis test with significance level $0.025$,
the critical value is $c=t_{n-1-q}^{-1}(1-0.025)$. If $T>c$, we
reject $H_{0}$. Here, $n$ is the sample size, $q$ is the length
of vector $Z_{it}$ (including the added intercept), and $t_{n-1-q}^{-1}(\gamma)$
denotes the $\gamma$-quantile of $t$-distribution with $(n-1-q)$
degrees of freedom. For example, suppose $Z_{it}$ includes two variables:
the previous day adherence $Y_{i,t-1}$ and the weather of day $t$.
Then $q=2+1=3$.

\textbf{Software.} R function to conduct primary hypothesis 2 is
\textsf{SARA\_primary\_hypothesis\_2} in \textsf{SARAanalysis.R}.

\subsubsection{Statistical details{*} \label{subsec:Technical-details-2}}

$\hat{\beta}$ in the test statistic (\ref{eq:test-primary2}) is
obtained by solving the following estimating equation (simutaneously
solving for $\hat{\alpha}$ and $\hat{\beta}$):
\begin{equation}
0=\sum_{i=1}^{n}\sum_{t=1}^{T_i - 1}S_{it} I_{it}e^{-A_{it}\beta}\left(Y_{i,t+1}-e^{Z_{it}^{T}\alpha+A_{it}\beta}\right)\begin{pmatrix}(A_{it}-p_{a})\\
e^{Z_{it}^{T}\alpha}Z_{it}
\end{pmatrix}.\label{eq:ee-primary2}
\end{equation}
Here, $p_{a}=P(A_{it}=1)$ is the randomization probability (assumed
constant across all time and all subjects). The superscript $T$ denotes
matrix transpose. Note that as in linear regression, we add an intercept
1 to the vector $Z_{it}$. It turns out that $\hat{\beta}$ is consistent
for $\beta_{0}$ in (\ref{eq:estimand-primary2}) even if $e^{Z_{it}^{T}\alpha}$
is a wrong model for $E(Y_{it}=1\mid A_{it}=0)$. In order words,
the choice of $Z_{it}$ only affects the variance of $\hat{\beta}$,
but does not affect the consistency of $\hat{\beta}$.

Mathematically, solving (\ref{eq:ee-primary2}) for $\hat{\beta}$
in this primary hypothesis is a special case of the estimating equation for the exploratory analysis
in Section \ref{subsec:Technical-details-3}. Therefore, the formula for
computing $\widehat{\text{Var}}(\hat{\beta})$ is deferred to Section
\ref{sec:var}.

\section{Statistical method for secondary analysis}

\subsection{Secondary hypothesis 1}

$H_{0}$: The 6pm reminder notification with an extra persuasive message
will not yield a higher rate of full completion of the survey or active
task the same day than not providing the extra persuasive message.

$H_{1}$: The 6pm reminder notification with an extra persuasive message
will not yield a higher rate of full completion of the survey or active
task the same day than not providing the extra persuasive message.

\textbf{Note:} the statistical method for testing this secondary hypothesis
is essentially the same as primary hypothesis 1 in Section \ref{subsec:Primary-hypothesis-1}.
The differences are: (i) the treatment is defined differently; (ii)
the variables allowed in $Z_{it}$ are information available prior
to 6pm on day $t$.

\subsection{Secondary hypothesis 2}

$H_{0}$: Among individuals who complete the active tasks, offering
a post-active-task-completion life-insight will not yield a higher
rate of the full completion of the survey or active task the next
day than not offering a life-insight after active tasks completion.

$H_{1}$: Among individuals who complete the active tasks, offering
a post-active-task-completion life-insight will yield a higher rate
of the full completion of the survey or active task the next day than
not offering a life-insight after active tasks completion.

\textbf{Note:} the statistical method for testing this secondary hypothesis
is essentially the same as primary hypothesis 2 in Section \ref{subsec:Primary-hypothesis-2}.
The differences are: (i) the treatment is defined differently; (ii)
the availability indicator $I_{it}=1$ if participant $i$ completed
the active tasks on day $t$, and $I_{it}=0$ otherwise.

\section{Statistical method for exploratory analysis}

\subsection{Exploratory analysis}

The following description of the exploratory analysis is an excerpt
from Mash's working paper.

``We plan to run exploratory analyses to examine how the effectiveness
of engagement strategies changes over time (we conjecture the effectiveness
will decrease). We will also run additional exploratory analysis to
assess effect moderation. We will examine how the effect of engagement
strategies is moderated by gender, weekdays/weekends, and whether
the day is Sunday vs other days of the week (we expect the longer
Sunday\textquoteright s daily surveys\textquoteright{} completion
may be lower than other days).''

% \textcolor{blue}{{[}SAM: Mash needs to rank order these potential moderators prior to seeing the data.{]}}

This section describes the statistical method for the second part
of the exploratory analysis, about effect moderation.

\subsubsection{Model and Estimator}

\textbf{Data notation for exploratory analysis.} The data collected
from participant $i$ is denoted by $(Z_{i1},A_{i1},Y_{i1},\ldots,Z_{i,T_i},A_{i,T_i},Y_{i,T_i})$,
where the subscript $1,2,\ldots,T_i$ denotes day in the study. $Y_{it}$
denotes the binary outcome (full completion of either the survey or
completion of both active tasks) on day $t$. $A_{it}$ is the indicator
of treatment on day $t$ ($A_{it}=1$ if sent 4pm push notification,
0 if not). $Z_{it}$ is the control variables on day $t$. The purpose
of $Z_{it}$ is to reduce variance in the analysis. $X_{it}$ is the
potential effect moderators on day $t$, which is a subset of $Z_{it}$.
For example, $X_{it}$ could include gender, whether day $t$ is a
weekday, and whether day $t$ is a Sunday. $Z_{it}$ could include
$X_{it}$ and some other covariates such as the weather of day $t$. Let $I_{it}$ denote the availability status indicator for day $t$: if unavailable, the treatment is not delivered.

\textbf{Moderated treatment effect on log scale.} Denote by $H_{it}$ the history
of subject $i$ up to day $t$ prior to $A_{it}$: $H_{it}=\{Z_{i1},A_{i1},Y_{i1},\ldots,Z_{i,t-1},A_{i,t-1},Y_{i,t-1},Z_{it}\}$.
To form the test statistic, we consider the moderated treatment effect
on log scale:

\begin{equation}
\log\left(\frac{E\{E[Y_{it}|A_{it}=1,H_{it}, I_{it} = 1]\mid X_{it}, I_{it} = 1\}}{E\{E[Y_{it}|A_{it}=0,H_{it}, I_{it} = 1]\mid X_{it}, I_{it} = 1\}}\right)=X_{it}^{T}\beta,\qquad\text{for all }t=1,2,\ldots,T_i.\label{eq:estimand-exploratory}
\end{equation}
The superscript $T$ denotes matrix transpose. As in usual linear
regression, we add an intercept 1 to the vector $X_{it}$. $\beta$
characterizes the magnitude of effect moderation, and it is a vector
of length $p$, where $p$ is the length of vector $X_{it}$ (including
the intercept). Notice that here we assume model (\ref{eq:estimand-exploratory})
holds for all $t=1,2,\ldots,T_i$; that is, the definition of $\beta$
in (\ref{eq:estimand-exploratory}) is not marginal across time, unlike
in (\ref{eq:estimand-primary1}) and (\ref{eq:estimand-primary2}).

\textbf{Estimator and standard error.} In Section \ref{subsec:Technical-details-3},
we describe how to construct an estimator $\hat{\beta}$ for this
moderated treatment effect $\beta$, and an estimate of its variance-covariance
matrix \textbf{$\widehat{\text{Var}}(\hat{\beta})$}. 
% If one wishes
%to test for $\beta=0$ (all entries of the vector $\beta$ being zero,
%which is analogous to an $F$-test in a multivariate linear regression),
%the test statistic $T$ is defined by
%\begin{equation}
%T=\hat{\beta}^{T}\{\widehat{\text{Var}}(\hat{\beta})\}^{-1}\hat{\beta}.\label{eq:estimand-exploratory}
%\end{equation}
%To conduct two-sided hypothesis test with significance level $0.025$,
%the critical value is
%\[
%c=F_{p,n-q-p}^{-1}\left(\frac{(n-q-p)(1-0.025)}{p(n-q-1)}\right).
%\]
%If $T>c$, we reject $H_{0}:\beta=0$. Here, $n$ is the sample size,
%$q$ is the length of vector $Z_{it}$ (including the added intercept),
%$p$ is the length of vector $X_{it}$ (including the added intercept),
%and $F_{p,n-q-p}^{-1}(\gamma)$ denotes the $\gamma$-quantile of
%$F$-distribution with $(p,n-q-p)$ degrees of freedom. 

\textbf{Software.} R function to conduct exploratory analysis is
\textsf{SARA\_exploratory\_analysis} in \textsf{SARAanalysis.R}.

\subsubsection{Statistical details{*} \label{subsec:Technical-details-3}}

$\hat{\beta}$ in the test statistic (\ref{eq:estimand-exploratory})
is obtained by solving the following estimating equation (simutaneously
solving for $\hat{\alpha}$ and $\hat{\beta}$):
\begin{equation}
0=\sum_{i=1}^{n}\sum_{t=1}^{T_i} I_{it} e^{-A_{it}X_{it}^{T}\beta}\left(Y_{it}-e^{Z_{it}^{T}\alpha+A_{it}X_{it}^{T}\beta}\right)\begin{pmatrix}(A_{it}-p_{a})X_{it}\\
e^{Z_{it}^{T}\alpha}Z_{it}
\end{pmatrix}.\label{eq:ee-exploratory}
\end{equation}
Here, $p_{a}=P(A_{it}=1)$ is the randomization probability (assumed
constant across all time and all subjects). Note that as in linear
regression, we add an intercept 1 to both the vector $X_{it}$ and
the vector $Z_{it}$. It turns out that $\hat{\beta}$ is consistent
for $\beta$ in (\ref{eq:estimand-exploratory}) even if $e^{Z_{it}^{T}\alpha}$
is a wrong model for $E(Y_{it}=1\mid A_{it}=0)$. In order words,
the choice of $Z_{it}$ only affects the variance of $\hat{\beta}$,
but does not affect the consistency of $\hat{\beta}$.

We present the computation of $\widehat{\text{Var}}(\hat{\beta})$ in the next section.

\section{Technical details on computing $\widehat{\text{Var}}(\hat{\beta})${*}}
\label{sec:var}

In this section, we describe the variance estimates $\widehat{\text{Var}}(\hat{\beta})$
in each of the hypothesis tests.

\subsection{General form}

Consider the following estimating equation, where $n$ denotes sample
size, and $T$ denotes the total number of time points (here the number
of days),
\begin{equation}
0=\sum_{i=1}^{n}\sum_{t=1}^{T}I_{it} e^{-A_{it}X_{it}^{T}\beta}\left(Y_{it}-e^{Z_{it}^{T}\alpha+A_{it}X_{it}^{T}\beta}\right)\begin{pmatrix}(A_{it}-p_{a})X_{it}\\
e^{Z_{it}^{T}\alpha}Z_{it}
\end{pmatrix}.\label{eq:ee-general}
\end{equation}
For $(\hat{\beta},\hat{\alpha})$ that solves (\ref{eq:ee-general}),
its estimated variance-covariance matrix $V$ equals
\begin{equation}
V=\frac{1}{n}M_{n}^{-1}\left\{ \frac{1}{n}\sum_{i=1}^{n}D_{i}^{T}(I_{i}-H_{i})^{-1}(Y_{i}-\mu_{i})(Y_{i}-\mu_{i})^{T}(I_{i}-H_{i}^{T})^{-1}D_{i}\right\} \left(M_{n}^{-1}\right)^{T}.\label{eq:V}
\end{equation}
The terms on the right hand side of (\ref{eq:V}) are defined as follows:
\[
M_{n}:=(1/n)\sum_{i=1}^{n}\sum_{t=1}^{T}\begin{bmatrix}-Y_{it}(A_{it}-p_{a})A_{it}e^{-A_{it}X_{it}^{T}\beta}X_{it}X_{it}^{T} & -(A_{it}-p_{a})e^{Z_{it}^{T}\alpha}X_{it}Z_{it}^{T}\\
-Y_{it}e^{-A_{it}X_{it}^{T}\beta}A_{it}e^{Z_{it}^{T}\alpha}Z_{it}X_{it}^{T} & Y_{it}e^{-A_{it}X_{it}^{T}\beta}e^{Z_{it}^{T}\alpha}Z_{it}Z_{it}^{T}-2e^{2Z_{it}^{T}\alpha}Z_{it}Z_{it}^{T}
\end{bmatrix},
\]
\[
D_{i}:=\begin{bmatrix}e^{-A_{i1}X_{i1}^{T}\beta}(A_{i1}-p_{a})X_{i1}^{T} & e^{-A_{i1}X_{i1}^{T}\beta}e^{Z_{i1}^{T}\alpha}Z_{i1}^{T}\\
\vdots & \vdots\\
e^{-A_{iT}X_{iT}^{T}\beta}(A_{iT}-p_{a})X_{iT}^{T} & e^{-A_{iT}X_{iT}^{T}\beta}e^{Z_{iT}^{T}\alpha}Z_{iT}^{T}
\end{bmatrix},
\]
\[
E_{i}:=\begin{bmatrix}-e^{Z_{i1}^{T}\alpha+A_{i1}X_{i1}^{T}\beta}A_{i1}X_{i1}^{T} & -e^{Z_{i1}^{T}\alpha+A_{i1}X_{i1}^{T}\beta}Z_{i1}^{T}\\
\vdots & \vdots\\
-e^{Z_{iT}^{T}\alpha+A_{iT}X_{iT}^{T}\beta}A_{iT}X_{i1}^{T} & -e^{Z_{iT}^{T}\alpha+A_{iT}X_{iT}^{T}\beta}Z_{iT}^{T}
\end{bmatrix},
\]
\[
H_{i}:=\frac{1}{n}E_{i}M_{n}^{-1}D_{i}^{T},
\]
\[
Y_{i}:=\begin{bmatrix}I_{i1} Y_{i1}\\
\vdots\\
I_{iT} Y_{iT}
\end{bmatrix},\qquad\mu_{i}:=\begin{bmatrix}I_{i1} e^{Z_{i1}^{T}\alpha+A_{i1}X_{i1}^{T}\beta}\\
\vdots\\
I_{iT} e^{Z_{iT}^{T}\alpha+A_{iT}X_{iT}^{T}\beta}
\end{bmatrix},
\]
and $I_{i}$ is a $T\times T$ identity matrix. The superscript $T$
denotes matrix transpose.

\subsection{For primary hypothesis 1}

The $\widehat{\text{Var}}(\hat{\beta})$ in the test statistic (\ref{eq:test-primary1})
of primary hypothesis 1 is obtained as follows. In (\ref{eq:ee-general}),
set $X_{it}=1$ for all $i,t$, and set $T=T_i$. With these modification,
(\ref{eq:ee-general}) becomes equivalent to the estimating equation
(\ref{eq:ee-primary1}) presented in Section \ref{subsec:Technical-details-1}.
Then the (1,1) entry of the matrix $V$ in (\ref{eq:V}) is the estimated
variance $\widehat{\text{Var}}(\hat{\beta})$ in (\ref{eq:test-primary1}).

\subsection{For primary hypothesis 2}

The $\widehat{\text{Var}}(\hat{\beta})$ in the test statistic (\ref{eq:test-primary2})
of primary hypothesis 2 is obtained as follows. In (\ref{eq:ee-general}),
set $I_{it} \leftarrow I_{it} S_{it}$, replace $Y_{it}$ by $Y_{i,t+1}$, set
$T= T_i - 1$. With these modification, (\ref{eq:ee-general}) becomes equivalent to the estimating equation (\ref{eq:ee-primary2})
presented in Section \ref{subsec:Technical-details-2}.
Then the (1,1) entry of the matrix $V$ in (\ref{eq:V}) is the estimated
variance $\widehat{\text{Var}}(\hat{\beta})$ in (\ref{eq:test-primary2}).

\subsection{For exploratory analysis}

The $\widehat{\text{Var}}(\hat{\beta})$ in the test statistic (\ref{eq:estimand-exploratory})
of exploratory analysis is obtained as follows. In (\ref{eq:ee-general}),
set $T=T_i$. Suppose $p$ is the length of vector $X_{it}$ (including
the added intercept). Then the $p$-th principal submatrix (i.e.,
the top-left $p\times p$ submatrix) of the matrix $V$ in (\ref{eq:V})
is the estimated variance-covariance matrix $\widehat{\text{Var}}(\hat{\beta})$
in (\ref{eq:estimand-exploratory}).
\end{document}
